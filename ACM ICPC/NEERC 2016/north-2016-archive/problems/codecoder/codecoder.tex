%%%%%%%%%%%%%%%%%%%%%%%%%%%%%%%%%%%%%%%%%%%%%%%%%%%%%%%%%%%%%%%%%%
% ACM ICPC 2016-2017, NEERC                                      %
% Northern Subregional Contest                                   %
% St Petersburg, October 22, 2016                                %
%%%%%%%%%%%%%%%%%%%%%%%%%%%%%%%%%%%%%%%%%%%%%%%%%%%%%%%%%%%%%%%%%%
% Problem C. CodeCoder vs TopForces                              %
%                                                                %
% Original idea         Egor Kulikov                             %
% Problem statement     Egor Kulikov                             %
% Test set              Egor Kulikov                             %
%%%%%%%%%%%%%%%%%%%%%%%%%%%%%%%%%%%%%%%%%%%%%%%%%%%%%%%%%%%%%%%%%%
% Problem statement                                              %
%                                                                %
% Author                Egor Kulikov                             %
%%%%%%%%%%%%%%%%%%%%%%%%%%%%%%%%%%%%%%%%%%%%%%%%%%%%%%%%%%%%%%%%%%

\begin{problem}{CodeCoder vs TopForces}{codecoder.in}{codecoder.out}{\timeLimit}

% Original idea : Egor Kulikov
% Text          : Egor Kulikov
% Tests         : Egor Kulikov

Competitive programming is very popular in Byteland. 
In fact, every Bytelandian citizen is registered at two programming sites~---
CodeCoder and TopForces. 
Each site maintains its own proprietary rating system. 
Each citizen has a unique integer rating at each site that approximates their skill.
Greater rating corresponds to better skill.

People of Byteland are naturally optimistic. 
Citizen $A$ thinks that he has a chance to beat citizen $B$ in a programming competition
if there exists a sequence of Bytelandian citizens $A = P_0, P_1, \ldots, P_k = B$
for some $k \geq 1$ such that for each $i$ ($0 \leq i < k$), $P_i$ has higher rating
than $P_{i + 1}$ at one or both sites.

Each Bytelandian citizen wants to know how many other citizens they can possibly beat 
in a programming competition.

\InputFile

The first line of the input contains an integer $n$~--- the number of
citizens ($1 \leq n \leq 100\,000$).
The following $n$ lines contain information about ratings. The $i$-th of them
contains two integers $CC_i$ and $TF_i$~--- ratings of the $i$-th 
citizen at CodeCoder and TopForces
($1 \leq CC_i, TF_i \leq 10^6$).
All the ratings at each site are distinct.

\OutputFile

For each citizen $i$ output an integer $b_i$~---
how many other citizens they can possibly beat in a programming competition.
Each $b_i$ should be printed in a separate line, in the order
the citizens are given in the input.

\Example

\begin{example}
\exmp{
4
2 3
3 2
1 1
4 5
}{
2
2
0
3
}%
\end{example}

\end{problem}

