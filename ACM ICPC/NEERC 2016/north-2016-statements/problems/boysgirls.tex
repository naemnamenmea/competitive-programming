\begin{problem}{Boys and Girls}{boysgirls.in}{boysgirls.out}{\timeLimit}

% Original idea : Pavel Mavrin
% Text          : Pavel Mavrin
% Tests         : Pavel Mavrin

Bob found a nice task in his old math book for children. It says:

\begin{displayquote}
There are 10 children standing in a circle, 5 of them stand next to a boy, and 7 of them stand next to a girl. How is it possible?
\end{displayquote}

Here is the solution to the task. If 4 boys and 6 girls stand like this: \texttt{BGBGBGBGGG},
there are 5 children who stand next to a boy
(here they are underlined: \texttt{B\underline{G}B\underline{G}B\underline{G}B\underline{G}G\underline{G}}),
and 7 children who stand next to a girl (\texttt{\underline{B}G\underline{B}G\underline{B}G\underline{B}\underline{G}\underline{G}\underline{G}}).

Now Bob wants to solve a generalized version of this task:

\begin{displayquote}
There are $n$ children standing in a circle, $x$ of them stand next to a boy, and $y$ of them stand next to a girl. How is it possible?
\end{displayquote}

Help Bob by writing a program that solves the generalized task.

\InputFile

The single line of the input contains three integers $n$, $x$ and $y$
($2 \leq n \leq 100\,000$; $0\leq x, y\leq n$).

\OutputFile

If there is a solution, output a string of length $n$, describing the order of children in the circle. Character \chr{G} corresponds to a girl, character \chr{B} corresponds to a boy.
If there are several solutions, output any of them.

If there is no solution, output \txt{Impossible}.

\Examples

\begin{example}
\exmp{
10 5 7
}{
BGBGBGBGGG
}%
\exmp{
10 3 8
}{
Impossible
}%
\end{example}

\end{problem}
