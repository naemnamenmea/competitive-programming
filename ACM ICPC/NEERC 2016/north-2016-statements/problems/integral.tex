\begin{problem}{Integral Polygons}{integral.in}{integral.out}{\timeLimit}

% Original idea : Niyaz Nigmatullin
% Text          : Dmitry Shtukenberg
% Tests         : Niyaz Nigmatullin

Ingrid holds a polygon shop in a far away country. She sells only convex
polygons with integer coordinates. Her customers
prefer polygons that can be cut into two halves \emph{in a proper way}, that is
the cut should be straight with starting and ending points in the polygon vertices
and both halves should be non-empty and have integer areas.
The more ways to cut the polygon in the proper way are~--- the more expensive the polygon is.

For example, there are three ways to cut the left polygon in the proper way,
and two ways for the right polygon.

\begin{center}
\begin{tabular}{cc}
    \includegraphics{pics/integral.1} &
    \includegraphics{pics/integral.2}
\end{tabular}
\end{center}

The polygons in the shop are always of excellent quality, so the business is expanding.
Now Ingrid needs some automatic tool to determine the number of ways to cut
the polygon in the proper way. This is very important for her shop, since otherwise you
will spend a lot of time on setting prices~--- just imagine how
much time would it take to set prices for a medium-sized van with polygons. 
Could you help Ingrid and write the tool for her?

\InputFile

The first line of the input contains an integer $n$~---
the number of polygon vertices ($4 \leq n \leq 200\,000$).
Each of the following $n$ lines contains vertex coordinates: 
a pair of integers $x_i$ and $y_i$ per line ($-10^9 \le x_i, y_i \le 10^9$).

The specified polygon is convex and its vertices are specified in the order of traversal.

\OutputFile

Output a single integer $w$~--- the number of ways to cut the polygon in the proper way.

\Example

\begin{example}
\exmp{
5
7 3
3 5
1 4
2 1
5 0
}{
3
}%
\exmp{
4
1 1
3 1
5 5
1 3
}{
2
}%
\end{example}

\end{problem}
