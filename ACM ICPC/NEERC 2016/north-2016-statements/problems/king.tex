\begin{problem}{King's Heir}{king.in}{king.out}{\timeLimit}

% Original idea : Andrew Stankevich
% Text          : Andrew Stankevich
% Tests         : Pavel Kunyavsky

The king is dead, long live the king! After the sudden death of the king 
Fert XIII the people of the Flatland Kingdom are going
to welcome the new king. Unfortunately, there is a problem,
Fert has too many sons. 

Actually, he has $n$ sons and he loved each new son more
than all of his previous sons. Well, probably he just stopped loving
his sons because of their bad behavior. Anyway, after the new son 
was born Fert made the new testament that declared that the newly born son
would be the heir.

However, there is a problem. Only the king's son who is at least
18 years old at the moment of the king's death can become a new king.
Now the ministers of the government are trying to find the correct
new king, but they seem to fail. Help them!

\InputFile

The first line of the input contains three integers: $d$, $m$ and $y$~---
the day, the month and the year of the king's death, $d$ is from 1 to 31,
$m$ is from 1 to 12, $y$ is from 1 to 9999. It is guaranteed that there
exists day $d$ in month $m$, all months have the same number of days
in Flatland as in our country, except that Flatland calendar doesn't have
leap years, so February (month 2) always has 28 days.

The second line contains $n$ ($1 \le n \le 100$)~--- the number of king's sons.
The following $n$ lines contain three integers each $d_i$, $m_i$ and $y_i$
and specify the birth dates of king's sons. All dates are correct and no son is born after or on the day of king's death.
The king had no twins, so 
no two sons were born on the same date.

\OutputFile

Output one integer~--- the number of the son that would become the king, or
$-1$ if none of them is at least 18 years old.
The sons are numbered from 1 to $n$ in order they are described in the input.
The youngest son who is at least 18 years old at the moment of the king's
death would become the king. If the son has his 18th birthday exactly
on the day of the king's death, he can become a king.

\Examples

\begin{example}
\exmp{
22 10 2016
7
28 2 1999
22 7 1995
21 10 1998
23 10 1998
3 9 2000
1 4 2013
17 12 2004
}{
3
}%
\exmp{
22 10 2016
1
28 2 1999
}{
-1
}%
\end{example}

\end{problem}
